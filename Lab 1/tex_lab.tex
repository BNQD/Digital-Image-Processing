% Options for packages loaded elsewhere
\PassOptionsToPackage{unicode}{hyperref}
\PassOptionsToPackage{hyphens}{url}
%
\documentclass[
]{article}
\usepackage{lmodern}
\usepackage{amssymb,amsmath}
\usepackage{ifxetex,ifluatex}
\ifnum 0\ifxetex 1\fi\ifluatex 1\fi=0 % if pdftex
  \usepackage[T1]{fontenc}
  \usepackage[utf8]{inputenc}
  \usepackage{textcomp} % provide euro and other symbols
\else % if luatex or xetex
  \usepackage{unicode-math}
  \defaultfontfeatures{Scale=MatchLowercase}
  \defaultfontfeatures[\rmfamily]{Ligatures=TeX,Scale=1}
\fi
% Use upquote if available, for straight quotes in verbatim environments
\IfFileExists{upquote.sty}{\usepackage{upquote}}{}
\IfFileExists{microtype.sty}{% use microtype if available
  \usepackage[]{microtype}
  \UseMicrotypeSet[protrusion]{basicmath} % disable protrusion for tt fonts
}{}
\makeatletter
\@ifundefined{KOMAClassName}{% if non-KOMA class
  \IfFileExists{parskip.sty}{%
    \usepackage{parskip}
  }{% else
    \setlength{\parindent}{0pt}
    \setlength{\parskip}{6pt plus 2pt minus 1pt}}
}{% if KOMA class
  \KOMAoptions{parskip=half}}
\makeatother
\usepackage{xcolor}
\IfFileExists{xurl.sty}{\usepackage{xurl}}{} % add URL line breaks if available
\IfFileExists{bookmark.sty}{\usepackage{bookmark}}{\usepackage{hyperref}}
\hypersetup{
  hidelinks,
  pdfcreator={LaTeX via pandoc}}
\urlstyle{same} % disable monospaced font for URLs
\usepackage{graphicx,grffile}
\makeatletter
\def\maxwidth{\ifdim\Gin@nat@width>\linewidth\linewidth\else\Gin@nat@width\fi}
\def\maxheight{\ifdim\Gin@nat@height>\textheight\textheight\else\Gin@nat@height\fi}
\makeatother
% Scale images if necessary, so that they will not overflow the page
% margins by default, and it is still possible to overwrite the defaults
% using explicit options in \includegraphics[width, height, ...]{}
\setkeys{Gin}{width=\maxwidth,height=\maxheight,keepaspectratio}
% Set default figure placement to htbp
\makeatletter
\def\fps@figure{htbp}
\makeatother
\setlength{\emergencystretch}{3em} % prevent overfull lines
\providecommand{\tightlist}{%
  \setlength{\itemsep}{0pt}\setlength{\parskip}{0pt}}
\setcounter{secnumdepth}{-\maxdimen} % remove section numbering

\date{}

\begin{document}

\textbf{Introduction}

This lab introduced various methods of image processing. Different
transforms will be applied to images and the effects will be examined.

\textbf{Section 2.1 - Q1}

\includegraphics{cid:Image_0.png}

\includegraphics{cid:Image_1.png}\includegraphics{cid:Image_2.png}

\textbf{Section 2.1 - Q2}

\includegraphics{cid:Image_3.png}

\includegraphics{cid:Image_4.png}

\textbf{Section 2.1 - Q3}

\includegraphics{cid:Image_5.png}

\includegraphics{cid:Image_6.png}

\textbf{Section 2.2 - Q1}

\includegraphics{cid:Image_7.png}

\includegraphics{cid:Image_8.png}\includegraphics{cid:Image_9.png}

\textbf{Section 2.2 - Q2}

\includegraphics{cid:Image_10.png}

\includegraphics{cid:Image_11.png}

\textbf{Section 2.2 - Q3}

\includegraphics{cid:Image_12.png}

\includegraphics{cid:Image_13.png}

\textbf{Section 2.2 - Q4}

\includegraphics{cid:Image_14.png}

\textbf{Section 2.2 - Q5}

\includegraphics{cid:Image_15.png}

\textbf{Section 2.2 - Q6}

\includegraphics{cid:Image_16.png}

\{0: 0, 1: 2, 2: 4, 3: 6, 4: 8, 5: 10, 6: 12, 7: 14, 8: 16, 9: 18, 10:
20, 11: 22, 12: 24, 13: 26, 14: 28, 15: 30, 16: 32, 17: 34, 18: 36, 19:
38, 20: 15, 21: 16, 22: 17, 23: 17, 24: 18, 25: 19, 26: 20, 27: 21, 28:
21, 29: 22, 30: 23, 31: 24, 32: 24, 33: 25, 34: 26, 35: 27, 36: 28, 37:
28, 38: 29, 39: 30, 40: 31, 41: 31, 42: 32, 43: 33, 44: 34, 45: 35, 46:
35, 47: 36, 48: 37, 49: 38, 50: 38, 51: 39, 52: 40, 53: 41, 54: 42, 55:
42, 56: 43, 57: 44, 58: 45, 59: 45, 60: 46, 61: 47, 62: 48, 63: 49, 64:
49, 65: 50, 66: 51, 67: 52, 68: 52, 69: 53, 70: 54, 71: 55, 72: 56, 73:
56, 74: 57, 75: 58, 76: 59, 77: 59, 78: 60, 79: 61, 80: 62, 81: 63, 82:
63, 83: 64, 84: 65, 85: 66, 86: 66, 87: 67, 88: 68, 89: 69, 90: 70, 91:
70, 92: 71, 93: 72, 94: 73, 95: 73, 96: 74, 97: 75, 98: 76, 99: 77, 100:
77, 101: 78, 102: 79, 103: 80, 104: 80, 105: 81, 106: 82, 107: 83, 108:
84, 109: 84, 110: 85, 111: 86, 112: 87, 113: 87, 114: 88, 115: 89, 116:
90, 117: 91, 118: 91, 119: 92, 120: 93, 121: 94, 122: 94, 123: 95, 124:
96, 125: 97, 126: 98, 127: 98, 128: 99, 129: 100, 130: 101, 131: 101,
132: 102, 133: 103, 134: 104, 135: 105, 136: 105, 137: 106, 138: 107,
139: 108, 140: 108, 141: 109, 142: 110, 143: 111, 144: 112, 145: 112,
146: 113, 147: 114, 148: 115, 149: 115, 150: 116, 151: 117, 152: 118,
153: 119, 154: 119, 155: 120, 156: 121, 157: 122, 158: 122, 159: 123,
160: 124, 161: 125, 162: 126, 163: 126, 164: 127, 165: 128, 166: 129,
167: 129, 168: 130, 169: 131, 170: 132, 171: 133, 172: 133, 173: 134,
174: 135, 175: 136, 176: 136, 177: 137, 178: 138, 179: 139, 180: 140,
181: 140, 182: 141, 183: 142, 184: 143, 185: 143, 186: 144, 187: 145,
188: 146, 189: 147, 190: 147, 191: 148, 192: 149, 193: 150, 194: 150,
195: 151, 196: 152, 197: 153, 198: 154, 199: 154, 200: 155, 201: 383,
202: 385, 203: 387, 204: 389, 205: 391, 206: 393, 207: 395, 208: 397,
209: 399, 210: 400, 211: 402, 212: 404, 213: 406, 214: 408, 215: 410,
216: 412, 217: 414, 218: 416, 219: 418, 220: 420, 221: 421, 222: 423,
223: 425, 224: 427, 225: 429, 226: 431, 227: 433, 228: 435, 229: 437,
230: 439, 231: 441, 232: 442, 233: 444, 234: 446, 235: 448, 236: 450,
237: 452, 238: 454, 239: 456, 240: 458, 241: 460, 242: 462, 243: 463,
244: 465, 245: 467, 246: 469, 247: 471, 248: 473, 249: 475, 250: 477,
251: 479, 252: 481, 253: 483, 254: 484, 255: 486\}

\textbf{Conclusion}

Throughout the lab, various forms of image processing were used and
analyzed.

\textbf{Analysis}

Question 1

To do an alpha mask you can let intensity

I(x,y) = (Ia(x,y) + Ib(x,y)*Normalize(M(x,y)))/2

- Where Ia is the intensity of image a

- Where Ib is the intensity of image b

- Where M is the intensity of the mask

This is taking the average of the two images intensities taking the mask
into account

Question 2

- T(r) applies a contrast\_stretch which normalizes the contrast in the
image. In other words evenly distributing the intensities over the range
255 to 0.

Question 3

- LUT can also improve performance for computation intensive transforms.

- The LUT takes alot of memory which can be bad if running on a small
machine.

- The LUT harder to modify than a function pointer.

- In the case of contrast stretching it depends on the image making it
non reusable.

- LUT's can't be reused for translation transforms

Question 4

- Contrast stretching could be used to bring the higher intensity range
down to the display range by normalizing all the values then multiplying
them by 2\^{}8.

- Ie 2\^{}8*(r/(r\_max-r\_min)-r\_min/(r\_max-r\_min)

- Some considerations should be made if there is large amounts of
intensities in a small range as the detail will be lost on the display.

Question 5

- If an image is very dim then the range of intensities would be low
making the contrast stretching hard to impossible. As many distinct
points would share intensities, they would be normalized to the same
value.

- Sensor noise on a dim image will have the effect of disabling or
significantly hindering the contrast stretch. This is due to the noise
widening the range between r\_min and r\_max. With the upper bound
having noise from 0 to 255 totally disabling the contrast stretch.

\end{document}
